Most supercomputers are represented as computational facilities of collective
use, providing access to computing resources on a competitive basis.
In these conditions an individual user or a project group, with a large quota
of allocated resources at a supercomputer, confronts the real task: what
strategy to choose in order to utilize provided quota successfully.
The term ``successfully'' is understood as an ability [for the user/group] to
utilize the entire allocated quota in the range of the requested time.
Resource utilization is an actual usage of a partial or full amount of
allocated resources within the time that is equal or less to the time for which
these resources were allocated, so $ResourceUtilization \leq ResourceAllocation$
($cores \times hours$).
First of all, the utilization process is affected by a dynamically changing
supercomputer load (i.e., number of busy nodes at a certain timestamp), by
competition for computing resources with other users and by the local policy of
a particular supercomputer which sets the rules for this competition.

Meanwhile, there is a set of job parameters that are under the user control,
called as variable parameters, used to launch jobs on the supercomputer, which
would eventually strongly influence the amount of computing resources consumed
during the requested time.
In the first place, such parameters include size and length of every job,
expressed in requested number of compute nodes and estimated execution time,
respectively, since the values of these parameters would affect jobs' waiting
time in the supercomputer's queue.
And this in turn will affect the total number of jobs that expected to be
launched on the supercomputer in a particular time period and, as a result, the
total amount of consumed resources.
Section~\ref{sec-strategy-1} provides the essential terminology
including descriptions with these variable parameters.

% A set of specific values of variable parameters defined by a specified user or
% a project group will be referred to as a job launch strategy for this user
% or group.

In some projects, not all variable parameters are allowed to be changed.
For example, there are projects that can not work with more than one compute
node, but in general, for other projects such choice is possible.
In this case, the task of choosing an outperform strategy for defining job
parameters and corresponding launching scheme becomes relevant.

% In this case, the task of choosing a strategy, which increases probability of
% successful utilization of a large allocated quota of supercomputer time,
% becomes relevant.

Finding the strategy that outperforms any arbitrary one is a complex task due
to dependence of the corresponding variable parameters on a great number of
dynamically varying factors that are difficult to predict because of other
users activities.
The task can be simplified by keeping values of variable parameters unchangeable
over a period of campaign running time (i.e., static strategy over a long period
of time).
% The task can be simplified by looking for a static execution strategy, which
% means a strategy with values of variable parameters that do not change over
% time.
Our hypothesis is that in most cases such strategy will give a higher
probability of successful utilization of a large allocated quota of
supercomputer's resources, for example, than a random dynamic strategy.

There is no only one such outperforming strategy that would work for all
supercomputers or will fit all users and project groups needs, since every
strategy depends on a particular computing workflow and needs of one who
requests this strategy.

Therefore, an effective utilization of allocated resources relies on
well-defined strategy, which should guarantee an achievement of the requested
utilization over defined time interval, thus such strategy will maximize the
probability of utilizing a particular allocated resources entirely.
It is also possible that the utilization of the whole allocation time is not
feasible, in this case the appropriate strategy will facilitate to increase the
utilization in compare to no strategy at all.
Furthermore, such approach is applicable for the estimation of probable
utilization value of potentially allocated resources according to defined job
launch parameters, supercomputer load, and chosen strategy.

This paper provides key points of design and development of an approach to and
technique of finding static strategies of jobs launch for a given project on
given supercomputer resources, which increases the probability of successful
utilization of a large allocated quota of computing resources.
