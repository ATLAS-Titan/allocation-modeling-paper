Designed modeling tools mentioned earlier include the quantitative model for
estimation of the probability of resources utilization with chosen execution
strategy and the simulator that provides a simulation of jobs being executed at
the defined resources (i.e., simulation of workflow's execution), where
resources availability is described by predefined workload based on either
historical log data or synthetic data (simulation of resources utilization for
defined set of jobs with chosen execution strategy).

The simulator provides resulting data (one run of a simulation) for plotting a
distribution of utilization over the simulation time according to the defined
resources workload and the execution strategy for observed jobs.
But this plot might be varying since the process of utilization depends on
random variables.
Thus, many simulations are needed to move from one random result to a
distribution of probabilities, but these are time expensive processes.
On the other hand, the quantitative model gives theoretical calculations on a
distribution of probabilities with defined resources workload and chosen
execution strategy, but it doesn't consider the effect of the applied strategy
on the resources workload.
As the result we have that the simulator provides parameters describing the
state of resources (e.g., workload considering the load from observed jobs with
the defined strategy) and job related characteristics (e.g., a probability of
a single job resource utilization) to the quantitative model to get the
probability estimation of the resources utilization.

\subsubsection{Quantitative Model}
\label{sec-strategy-2-1}

The quantitative model allows to calculate a probability of achieving
required utilization of allocated resources in a given time by defining
specific parameters for jobs processing.
Some of these parameters are set by the user (e.g., the number of requested
nodes and wall time), while other parameters are determined by a particular
workload of a supercomputer and actions (i.e., activity) of other users (e.g.,
the waiting time of the job in the supercomputer's queue before it runs on
compute nodes).
The basic version of the model assumes that jobs, which utilization is under the
estimation, are launched sequentially: every next job arrives to the
supercomputer queue only after the previous job has been started to run on
compute nodes.

Equation that describes the quantitative model (its derivation process is
presented in the Appendix~\ref{appendix-model-derivation}):
\begin{equation}
    \label{eq-quantitative-model}
    \begin{multlined}
    P(U > U_0) = \sum\limits_{n=100}^{\infty} 
                 \bigg[
                 \int_{U_0}^{\infty}f(x, n\mu_{U}, n\sigma_{U}^2)dx \
                 \times \\
                 \bigg( \int_{-\infty}^{T_0}f(x, n\mu, n\sigma^2)dx \  - \\
                 \int_{-\infty}^{T_0}f(x, (n+1)\mu, (n+1)\sigma^2)dx \bigg)
                 \bigg]
    \end{multlined}
\end{equation}

The outcome of the Equation~\ref{eq-quantitative-model} is the probability
that the utilization of resources, which is achieved by a sequential set of
processed jobs using capabilities of the supercomputer during the time
interval $T_0$, is greater than the predefined value $U_0$. It implies that:
\begin{itemize}
    \item Utilization of every single job is described by a random variable
    with the expected value $\mu_{U}$ and the variance $\sigma_{U}^2$;
    \item The time interval between launches of sequential jobs is described
    by a random variable with the expected value $\mu$ and the
    variance $\sigma^2$.
\end{itemize}

\subsubsection{Simulator}
\label{sec-strategy-2-2}

Another modeling tool is the simulator which is aimed to simulate the workload
on a supercomputer and to produce job traces for a given workload, as well,
it is used for the quantitative model validation and adjustment.
There are two modes to run the simulator:
i) set operational parameters, such as job generation rate and job execution
rate, to produce synthetic data only;
ii) use historical data for key parameters of a job, such as timestamp of job
arrival to the queue and job execution time, to produce simulated data of
real job processing life-cycle.
(Detailed description of the simulator is presented in the
Appendix~\ref{appendix-simulator-description}.)
