This work aims to modulate resource supply with allocation, while many other
papers have investigated how to shape resources request in response to
resource supply. Thus in principle, this work can be both useful for long
running campaigns (where allocation means the money spent) as well as need
to extend it. In terms of the conducted research developed tools provide
possibility to adjust the campaign execution strategy that would improve the
probability of utilizing a given allocation for a given project. The
essential regulations are based on forming outperformed job parameters.

Further work for the proposed approach improvement includes reinforcement of
applied requirements (i.e., decrease the number of applied assumptions for
the developed tools). The quantitative model and simulator are preliminary
and require further tuning, to understand the accuracy and sensitivity (to
initial conditions, training duration, workload types). This early work will
be extended to consider different kinds of workflows as well as different
types of workloads of heterogeneous resources.
