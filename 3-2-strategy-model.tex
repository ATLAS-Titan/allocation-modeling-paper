The developed model allows to calculate the probability of achieving the required utilization of allocated resources in a given time by defining specific parameters for jobs processing. Some of these parameters are set by the user (e.g., the number of requested nodes and walltime), while other parameters are determined by the workload of the supercomputer and the actions (i.e., activity) of other users (e.g., the waiting time of the job in the supercomputer's queue before it runs on computing nodes).

The base version of the model assumes that jobs, which utilization is under the estimation, are launched sequentially: every next job arrives to the supercomputer queue only after the previous job has been started to run on computing nodes. Also, the model can be adapted to other schemes of jobs launching. For example, if jobs are launched sequentially according to the scheme that ``the every next job enters the queue only after the previous job left it'', then the basic model can be used with ``the virtual waiting time of the job'', which equals to the sum of their real waiting time and their real execution time. If the launching scheme assumes several input streams, e.g., 2-3 streams, then the basic model with one stream can be used, but the defined time for calculated utilization will be reduced by 2-3 times respectively. A formal description of the input and output data for the basic model is presented below.

\textit{Given assumptions}:
\begin{itemize}
    \item Jobs $J$ of project $Pr$;
    \item Jobs $J$ require $N$ nodes, where $N$ is a random variable with expected value $\mu_{N}$ and variance $\sigma_{N}^2$;
    \item Jobs $J$ require walltime $E$, where $E$ is a random variable with expected value $\mu_{E}$ and variance $\sigma_{E}^2$;
    \item Execution times of jobs $J$ equal to their walltime values;
    \item Duration of waiting time in the queue for jobs $J$ is described by a random variable $Q$ with expected value $\mu$ and variance $\sigma^2$;
    \item Jobs $J$ come into the supercomputer queue sequentially: the next job is allocated to the queue after the previous one has left the queue to computing nodes.
\end{itemize}

\textit{Values to find}:
$P(U > U_0)$ - the probability that utilization $U$ during the time interval $T_0$ will exceed the predefined value $U_0$, where $T_0$ is big.

The derivation process is presented in the appendices (Appendix~\ref{appendix-model-derivation}), and here is the final equation that describes the quantitative model:
\begin{equation}
    \label{eq-quantitative-model}
    \begin{multlined}
    P(U > U_0) = \sum\limits_{n=100}^{\infty} 
                 \bigg[ \int_{U_0}^{\infty}f(x, n\mu_{U}, n\sigma_{U}^2)dx \  \times \\
                 \bigg( \int_{-\infty}^{T_0}f(x, n\mu, n\sigma^2)dx \  - \\
                 \int_{-\infty}^{T_0}f(x, (n+1)\mu, (n+1)\sigma^2)dx \bigg) \bigg]
    \end{multlined}
\end{equation}

The outcome of the Equation~\ref{eq-quantitative-model} is the probability that the utilization of resources, which is achieved by a sequential set of processed jobs using capabilities of the supercomputer during the time interval $T_0$, is greater than the predefined value $U_0$. It implies that:
\begin{itemize}
	\item Utilization of every single job is described by a random variable with the expected value $\mu_{U}$ and the variance $\sigma_{U}^2$;
	\item The time interval between launches of sequential jobs is described by a random variable with the expected value $\mu$ and the variance $\sigma^2$.
\end{itemize}
