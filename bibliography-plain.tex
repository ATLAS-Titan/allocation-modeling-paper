\bibitem{ref-qespera}
Murali, P., Vadhiyar, S.: Qespera: an adaptive framework for prediction of queue waiting times in supercomputer systems. Concurrency Computat.: Pract. Exper. \textbf{28}(9), 2685--2710 (2016) doi:10.1002/cpe.3735

\bibitem{ref-qbets}
Nurmi, D., Brevik, J., Wolski, R.: QBETS: Queue Bounds Estimation from Time Series. In: Frachtenberg, E., Schwiegelshohn, U. (eds.) Job Scheduling Strategies for Parallel Processing, Lecture Notes in Computer Science \textbf{4942}, pp.76--101. Springer, Berlin, Heidelberg (2008) doi:10.1007/978-3-540-78699-3\_5

\bibitem{ref-yang}
Yang, L.T., Ma, X., Mueller, F.: Cross-Platform Performance Prediction of Parallel Applications Using Partial Execution. In: Proceedings of the ACM/IEEE 2005 Supercomputing Conference, SC'05, pp.111--120 (2005) doi:10.1109/SC.2005.20

\bibitem{ref-guo}
Guo, J., Nomura, A., Barton, R., Zhang, H., Matsuoka, S.: Machine Learning Predictions for Underestimation of Job Runtime on HPC System. In: Yokota, R., Wu, W. (eds.) Supercomputing Frontiers, Lecture Notes in Computer Science, \textbf{10776}, pp.179--198. Springer, Cham (2018) doi:10.1007/978-3-319-69953-0\_11

\bibitem{ref-lerida}
Lérida, J.L., Solsona, F., Giné, F., Hanzich, M., García, J.R., Hernández, P.: MetaLoRaS: A Re-scheduling and Prediction MetaScheduler for Non-dedicated Multiclusters. In: Cappello, F., Herault, T., Dongarra, J. (eds.) Recent Advances in Parallel Virtual Machine and Message Passing Interface, Lecture Notes in Computer Science, \textbf{4757}, pp.195--203. Springer, Berlin, Heidelberg (2007) doi:10.1007/978-3-540-75416-9\_30

\bibitem{ref-sotiriadis}
Sotiriadis, S., Bessis, N., Antonopoulos, N.: Towards Inter-cloud Schedulers: A Survey of Meta-scheduling Approaches. In: 6th IEEE International Conference on P2P, Parallel, Grid, Cloud and Internet Computing, pp.59--66 (2011) doi:10.1109/3PGCIC.2011.19

\bibitem{ref-legrand}
Legrand, A., Trystram, D., Zrigui, S.: Adapting Batch Scheduling to Workload Characteristics: What can we expect From Online Learning? In: 33rd IEEE International Parallel \& Distributed Processing Symposium (IPDPS), pp.1--10 (2019)

\bibitem{ref-queueing-theory}
Cooper, R.B.: Introduction to Queueing Theory, 2nd Ed. Elsevier/North-Holland (1981)

\bibitem{ref-kendall}
Kendall, D.G.: Some Problems in the Theory of Queues. J. Royal Stat. Soc. Ser. B Methodol. \textbf{13}(2), 151--185 (1951)

\bibitem{ref-qss}
Titov, M., et al.: Queueing System Simulator (QSS) [software], \url{https://github.com/ATLAS-Titan/allocation-modeling} [accessed on 2019-04-15]

\bibitem{ref-panda}
Barreiro Megino, F.H., De, K., Klimentov, A., Maeno, T., Nilsson, P., Oleynik, D., Padolski, S., Panitkin, S., Wenaus, T.: PanDA for ATLAS distributed computing in the next decade. J. Phys. Conf. Ser. \textbf{898}(5), 052002 (2017) doi:10.1088/1742-6596/898/5/052002

\bibitem{ref-atlas}
ATLAS Collaboration: The ATLAS Experiment at the CERN Large Hadron Collider. JINST \textbf{3}, S08003 (2008)

\bibitem{ref-titan-prodsys}
Barreiro Megino, F.H., De, K., Jha, S., Klimentov, A., Maeno, T., Nilsson, P., Oleynik, D., Padolski, S., Panitkin, S., Wells, J., Wenaus, T.: Integration of Titan supercomputer at OLCF with ATLAS Production System. J. Phys. Conf. Ser. \textbf{898}(9), 092002 (2017) doi:10.1088/1742-6596/898/9/092002

\bibitem{ref-saga}
Goodale, T., Jha, S., Kaiser, H., Kielmann, T., Kleijer, P., Von Laszewski, G., Lee, C.A., Merzky, A., Rajic, H.L., Shalf, J.: SAGA: A Simple API for Grid Applications. High-level application programming on the Grid. Comput. Methods Sci. Technol. \textbf{12}(1), 7--20 (2006) doi:10.12921/cmst.2006.12.01.07-20
